Om de computer een keuze te laten maken gebruiken we het key-woord \texttt{if}\index{if}. Bijvoorbeeld:
\begin{lstlisting}[language=python]
if a < 5:
    a += 1
\end{lstlisting}
Hier staat: \textquote{als a kleiner is dan 5 tel dan 1 op bij a}. In alle andere gevallen doet de computer niets. We hebben de computer dus een simpele berekening laten uitvoeren op basis van de conditie of \texttt{a} kleiner dan 5 is. \texttt{a < 5} is de conditie. \texttt{if} geeft aan dat er een conditie komt. \texttt{:} zegt dat als de condite waar is dat dan de inspringende code uitgevoerd moet worden.

De conditie kent twee waarden: True\index{True} en False\index{False}. Let op binnen Python worden beide met de eerste letter als hoofdletter geschreven. We kunnen dan ook code maken met een True:
\begin{lstlisting}[language=python]
if True:
    a += 1
\end{lstlisting}
Deze \texttt{if}-conditie is altijd waar en dus zal altijd de \texttt{a} plus 1 uitgevoerd worden. Zinloze code dus, maar het laat wel zien dat een conditie waar of onwaar moet zijn.

De afspraak (syntax\index{syntax}) binnen Python is dat code met spaties moet inspringen\index{inspringen} na een conditie. Alle inspringende code behoort bij de uit te voeren code (code block\index{code block} na de \texttt{if}. Als het inspringen stopt gaat Python ervan uit dat het \texttt{if}-statement voorbij is. De keuze voor het aantal spaties is geheel naar de keuze van de programmeur, hoewel er veel gebruik wordt gemaakt van 4 spaties. Ook wij zullen deze ongeschreven regel volgen en altijd 4 spaties gebruiken bij het inspringen.
\begin{lstlisting}[language=python]
if a < 5:
    a += 1
print(a)
\end{lstlisting}
Het \texttt{print} commando wordt niet meer ingesprongen en behoort dus niet bij de \texttt{if}. De \texttt{print} wordt dus altijd uitgevoerd, ook als \texttt{a} groter is dan 5.

Elke verandering van het inspringen betekent voor Python dat er een nieuw code block begint. Een code block moet dan ook altijd met dezelfde hoeveelheid spaties inspringen.
\begin{lstlisting}[language=python]
if a < 5:
    a += 1
    print(a)
\end{lstlisting}
Zowel de regel met het 1 optellen bij \texttt{a} als de regel met het print-statement springen beide 4 spaties in en horen dus bij elkaar, ze vormen hetzelfde code block.

Als we code laten inspringen zonder dat er een statement is of we springen binnen een code block in met een ongelijk aantal spaties dan krijgen we een syntax-error van Python. Deze code:
\begin{lstlisting}[language=python]
a = 1
b = 2
   c =  a+b
print(c)
\end{lstlisting}
geeft bij het uitvoeren deze error:
\begin{lstlisting}[language=python]
IndentationError: unexpected indent
\end{lstlisting}
Er is dus een \textquote{identation}\index{identation} fout, indentation is het Engelse woord voor inspringen. Deze error is een syntax error, omdat hij tegen de afspraken (syntax) van Python in gaat.

