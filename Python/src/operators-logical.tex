Met logische operators\index{logische operators} (logical operators\index{logical operator}) kunnen we twee condities tegelijk testen. Bij deze test is het geheel waar (True) of niet-waar (False). In de tabel \ref{table:logicop} vind je een overzicht van hoe om te gaan met logical operators.
\begin{flushleft}
\begin{table}[h!]
\centering
\begin{tabularx}{\textwidth}{ |c|X|c| }
\hline
	Operator &
	Betekenis &
	Hoe te gebruiken \\
\hline
	and\index{and}\index{operator!and} &
	Als beide condities waar zijn, dan is het hele statement waar (True) &
	x \textless 5 and y == 10 \\
\hline
	or\index{or}\index{operator!or} &
	Als \'e\'en van de condities waar is, dan is het hele statement waar (True) &
	x \textless 5 or y == 10 \\
\hline
\end{tabularx}
\caption{Logical operators}
\label{table:logicop}
\end{table}
\end{flushleft}

Met logische operator \texttt{not}\index{not}\index{operator!not} kunnen we een False conditie True maken of een True conditie False.
\begin{lstlisting}[language=python]
not(a == 5)
\end{lstlisting}
is dus hetzelfde als
\begin{lstlisting}[language=python]
a != 5
\end{lstlisting}
In combinatie met de \texttt{and} en de \texttt{or} wordt het dan:
\begin{lstlisting}[language=python]
not(x < 5 and y == 10)
\end{lstlisting}
Deze code betekent dat als \texttt{x} kleiner is dan 5 en \texttt{y} is 10 dan is de uitkomst \textbf{False}.

