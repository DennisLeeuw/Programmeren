De ASCII-tabel is \'e\'en van de oudste en was een van de meest gebruikte conversie-tabellen, zolang als computers voornamelijk door Engelstaligen werden gebruikt. Met de karakters van bijvoorbeeld de Europese of Slavische talen werd het allemaal wat lastiger en was het niet mogelijk om al deze karakters te vangen in 7-bits. Er ontstonden dan ook nadere conversie tabellen zoals UTF (Unicode Transformation Format) in 8-bits, 16-bits of zelfs 32-bits. Andere standaarden zijn de ISO 8859-1 en de Windows-1251 (ANSI Latin 1 of ANSI) standaard. Als we ook de Aziatische karakters willen kunnen weergeven dan redden we het ook niet met 8-bits, daar hebben we de 16 en 32-bits varianten voor.

