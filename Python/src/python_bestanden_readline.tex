Als eerste is er \texttt{readline} om een bestand regel voor regel te lezen. Het leest \'e\'en regel van een bestand. Een tweede readline opdracht leest de volgende regel:
\lstinputlisting[language=python]{code/bestand_readline.py}
Geven we meer \texttt{readline} opdrachten dan er regels in een bestand zitten dan zal python geen error geven, maar extra lege regels geven.

We kunnen ook met een \texttt{for} loop data uit een bestand lezen:
\lstinputlisting[language=python]{code/bestand_forline.py}

Als we met de data willen werken is het misschien handiger als elke regel uit een bestand als een item in een \textbf{list} terecht komt. Zo kunnen we per regel met het bestand werken:
\lstinputlisting[language=python]{code/bestand_readlines.py}

