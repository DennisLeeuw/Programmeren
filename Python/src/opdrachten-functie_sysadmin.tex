Opdrachten
\begin{enumerate}
\item Schijfruimte controleren
	\begin{description}
	\item[Doel] Werken met functies en modules.
	\item[Opdracht] Schrijf een functie \textquote{controleer\_schijfruimte()} die controleert hoeveel vrije ruimte er beschikbaar is op de C-schijf en een waarschuwing geeft als dit minder dan 5 GB is.
	\item[Tip] Gebruik de module \textquote{shutil} en de functie \textquote{disk\_usage()}
	\item[Voorbeeldoutput] output naar de gebruiker:
\begin{lstlisting}[language=python]
De vrije ruimte op C:\ is 10.4 GB - voldoende
\end{lstlisting}
of:
\begin{lstlisting}[language=python]
Waarschuwing: minder dan 5 GB beschikbaar!
\end{lstlisting}
	\end{description}

\item Automatisch gebruikers aanmaken
	\begin{description}
	\item[Doel] Functies met parameters schrijven.
	\item[Opdracht] Maak een functie \textquote{maak\_gebruiker\_aan(gebruikersnaam)} die een nieuwe gebruiker zou kunnen aanmaken (simulatie, geen echte systeemverandering, dus print wat er gedaan zou zijn).
	\item[Uitbreiding] Laat de functie een bevestiging printen, zoals:
\begin{lstlisting}[language=python]
Gebruiker 'student123' succesvol aangemaakt.
\end{lstlisting}
	\item[Bonus] Voeg een standaard wachtwoord toe als extra parameter.
	\end{description}

\item Logbestanden opschonen
	\begin{description}
	\item[Doel] Werken met bestandsbeheer via functies.
	\item[Opdracht] Schrijf een functie \textquote{verwijder\_oude\_logs(pad)} die alle bestanden met '.log' in een opgegeven map verwijderd die ouder zijn dan 30 dagen.
	\item[Tip] Gebruik \textquote{os}, \textquote{os.path} en \textquote{datetime}
	\item[Voorbeeldinput] De functie call zou er zo uit kunnen zien:
\begin{lstlisting}[language=python]
verwijder_oude_logs("C:/logs")
\end{lstlisting}
	\end{description}

\item Systeemrapport genereren (informatie)
	\begin{description}
	\item[Doel] Functies combineren en rapporteren.
	\item[Opdracht] Maak een functie \textquote{genereer\_systeemrapport()} die:
		\begin{itemize}
		\item de hoeveelheid RAM laat zien,
		\item de CPU-info toont,
		\item het IP-adres weergeeft.
		\end{itemize}
	\item[Tip] Gebruik modules zoals \textquote{psutil} en \textquote{socket}
	\item[Voorbeeldoutput] De output zou er zo uit kunnen zien:
\begin{lstlisting}[language=python]
Systeemrapport:
RAM: 8 GB
CPU: Intel Core i5
IP-adres: 192.168.1.101
\end{lstlisting}
	\end{description}
\end{enumerate}

\begin{enumerate}
\item Systeemrapport genereren (monitoring)
	\begin{description}
	\item[Doel] Actuele informatie opvragen met een functie
	\item[Opdracht] Maak een functie \textquote{genereer\_systeemrapport()} die:
		\begin{itemize}
		\item de hoeveelheid vrij beschikbare RAM laat zien,
		\item de huidige CPU-frequentie toont.
		\end{itemize}
	\item[Tip] Gebruik module \textquote{psutil}
	\item[Voorbeeldoutput] De output zou er zo uit kunnen zien:
\begin{lstlisting}[language=python]
Systeemrapport:
RAM beschikbaar: 6 GB
CPU frequentie: 2,56 MHz
\end{lstlisting}
	\end{description}
\end{enumerate}

