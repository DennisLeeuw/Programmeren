We gaan de \texttt{main.py} aanpassen zodat we de volgende functionaliteit krijgen:
\begin{enumerate}
\item We willen opdrachten per operating system aan de server kunnen geven. Voorlopig ondersteunen we alleen Windows:
	\begin{itemize}
	\item \url{http://127.0.0.1/windows?<opdracht>=<functie>}
	\end{itemize}
\item We gaan voor deze opdrachten werken met de uitkomsten van het PowerShell commando:
\begin{lstlisting}[language=bash]
Get-ItemProperty -Path "HKLM:\SOFTWARE\Microsoft\Windows NT\CurrentVersion"
\end{lstlisting}
Via de REST-API vragen we de gegevens op.
	\begin{itemize}
	\item \url{http://127.0.0.1/windows?version=CurrentBuild}
	\item \url{http://127.0.0.1/windows?version=DisplayVersion}
	\item \url{http://127.0.0.1/windows?version=ProductName}
	\end{itemize}
\end{enumerate}

De opdrachten:
\begin{enumerate}
\item Schrijf een PowerShell script dat via een commandline optie de verschillende stukjes informatie produceert. Zoek zelf uit hoe je opties meegeeft aan een PowerShell script.
\item Herschrijf daarna de Python \texttt{main.py} met de functies die de gevraagde informatie via REST in JSON produceert. Gebruik de subprocess module om het PowerShell script aan te roepen.
\end{enumerate}

