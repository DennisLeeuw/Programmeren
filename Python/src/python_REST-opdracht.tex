We gaan de \texttt{main.py} aanpassen zodat we de volgende functionaliteit krijgen:
\begin{enumerate}
\item We willen opdrachten per operating system aan de server kunnen geven. Voorlopig ondersteunen we alleen Windows:
	\begin{itemize}
	\item \url{http://127.0.0.1/windows?<opdracht>=<functie>}
	\end{itemize}
\item We gaan voor deze opdrachten werken met de uitkomsten van het PowerShell commando dat we geschreven hebben bij de PowerShell argumenten opdracht.
\item Via de REST-API vragen we de gegevens op.
	\begin{itemize}
	\item \url{http://127.0.0.1/windows?version=CurrentBuild}
	\item \url{http://127.0.0.1/windows?version=DisplayVersion}
	\item \url{http://127.0.0.1/windows?version=ProductName}
	\end{itemize}
\item Herschrijf de Python \texttt{main.py} met functies die de gevraagde informatie via REST in JSON produceert. Gebruik de subprocess module om het PowerShell script aan te roepen.
\end{enumerate}

