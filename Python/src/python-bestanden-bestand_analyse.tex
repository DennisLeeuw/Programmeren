\begin{itemize}
\item Leerdoelen
	\begin{enumerate}
	\item Extra informatie uit bestanden halen (bijvoorbeeld aantal regels in een .txt).
	\item Functies schrijven die herbruikbaar zijn.
	\item Het hele project integreren.
	\item Eindproduct presenteren en testen.
	\end{enumerate}

\item Inhoud
	\begin{enumerate}
	\item Combinatie van eerdere technieken.
	\item Uitbreiding met bestandsanalyse (bv. tellen van woorden/regels, bestandstype detecteren).
	\item Kleine debugging sessie.
	\item Documentatie schrijven (README.md).
	\item Optioneel: simpele menustructuur of command line interface toevoegen.
	\end{enumerate}

\item Opdracht: Breid het programma uit met een analysefunctie die per bestand extra info toevoegt (bv. voor .txt aantal regels, voor .csv aantal rijen).
\item Opdracht: Het complete “Bestanden-manager” programma opleveren dat:
	\begin{enumerate}
	\item Bestanden in een map kan uitlezen.
	\item Overzicht geeft in TXT, CSV en JSON.
	\item Bestanden analyseert (grootte, type, extra kenmerken).
	\item De gebruiker een simpele interface biedt.
	\end{enumerate}
\end{itemize}

