Python is een programmeertaal\index{programmeertaal}. Met een programmeertaal kan je een computer dingen voor je laten doen. Doordat de computer dan werk voor je doet kan jij andere dingen doen (nee, niet gamen). Tijdens het programmeren\index{programmeren} gebruik je een programmeertaal om programma-code\index{programma-code} of kortweg code\index{code} te schrijven. De programma-code schrijf je in een tekst-bestand zonder opmaak, we gebruiken daarom geen Word om een computer programma te schrijven maar een editor\index{editor}. Een editor is een heel simpele tekstverwerker zonder alle speciale functies als het invoegen van plaatjes, het nummeren van pagina's of het maken van hoofdstuk-koppen. De meeste programmeurs gebruiken een IDE\index{IDE} (Integrated Development Environment\index{Integrated Development Environment}).

Er zijn verschillende talen om een computer te vertellen wat hij moet doen. Python is er daar \'e\'en van. Programmeertalen kunnen grofweg in twee soorten verdeeld worden. We hebben de scripting-talen\index{scripting} en de talen die eerst gecompileerd\index{gecompileerd} moeten worden.

Talen die eerst gecompileerd moeten worden worden ook geschreven in een editor of een IDE, maar nadat het programma geschreven is gaat deze eerst door een compiler\index{compiler}. Deze compiler maakt van de programma-code een binary\index{binary}. Op Windows systemen is dit vaak een bestand met de extensie .exe van executable\index{executable}. Een executable is een programma dat direct door een computer uitgevoerd kan worden. De binary bevat 1-en en 0-en die direct door de computer begrepen worden. Een executable kan dan ook op elke computer, met dezelfde processor en hetzelfde operating system, uitgevoerd worden zonder dat de programma-code aanwezig is.

Scripting talen hebben geen compiler, maar een interpreter\index{interpreter}. De taal waarin het programma is geschreven wordt op de computer door de interpreter omgezet naar machinetaal\index{machinetaal}, of wel de 1-en en 0-en, om dan uitgevoerd te worden. Op elke computer waarop je het programma wilt draaien moet dus een interpreter en de code aanwezig zijn om de code te kunnen uitvoeren.

Python is een scriptingtaal en heeft dus een Python-interpreter nodig om gebruikt te kunnen worden.
