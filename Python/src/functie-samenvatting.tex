Functies kunnen om een aantal verschillende redenen gebruikt worden:
\begin{description}
\item [Onderhoud] Wijzigingen in de code hoeven maar in \'e\'en functie gedaan te worden, in plaats van op verschillende plekken in het programma waar dezelfde code gebruikt wordt
\item [Herbruikbaarheid] Je kunt dezelfde functie meerdere keren in een programma gebruiken (met bijvoorbeeld verschillende argumenten).
\item [Leesbaarheid] Door stukken code in functies te stoppen kan het hoofdprogramma beter leesbaar worden. Kleine specialistische functies zijn makkelijker te lezen en omdat ook het hoofdprogramma korter wordt is ook dat makkelijker leesbaar.
\end{description}

Een functie heeft een naam en kan voorzien worden van data via argumenten die meegeven worden aan parameters.

De functie kan data terug geven via de return-value.

