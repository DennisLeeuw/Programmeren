Modules kunnen hun eigen exceptions hebben. Een exception uit een module heeft de module naam als voorvoegsel:
\begin{lstlisting}[language=python]
#----------------#
# Import modules #
#----------------#
try:
    import requests
except ModuleNotFoundError:
    exit("De 'requests' module is niet aanwezig op dit systeem")

#-------------------------------#
# Variabelen voor ons programma #
#-------------------------------#
# Tegen welke host en port gaan we testen
url = "http://127.0.0.1"
# Header met domain name
headers = {
"Host: localhost"
}
# Timeout bij geen reactie
timeout = 5

#----------------------------------#
# Test de webserver op domain name #
#----------------------------------#
try:
    response = requests.get(url, headers=headers, timeout=timeout)
    if response.status_code == 200:
        print(f"De server op {ip} reageert op domein '{domain}' (HTTP {response.status_code}).")
    else:
        print(f"De server op {ip} reageert, maar gaf status {response.status_code} voor domein '{domain}'.")
except requests.ConnectionError:
    print(f"Geen verbinding mogelijk met {ip}.")
except requests.Timeout:
    print(f"Timeout bij het verbinden met {ip}.")
except requests.RequestException as e:
    print(f"Er trad een fout op: {e}")
\end{lstlisting}
Hier zien we dat de requests module zeker drie exceptions kent: \texttt{ConnectionError}, \texttt{Timeout} en de generieke error \texttt{RequestException}.
