We hebben gezien dat we in Python zelf programma's kunnen schrijven en dat we functies kunnen gebruiken om eenmaal geschreven code te hergebruiken. Het idee zou nu ontstaan kunnen zijn dat het ook mogelijk zou moeten zijn om functies van iemand anders te gebruiken. Dat kan natuurlijk door te knippen en plakken, maar kan dat ook makkelijker? Ja, dat kan door code van anderen te downloaden op ons systeem en deze vervolgens te gebruiken. Het beheer (downloaden, installeren en eventueel weer verwijderen) doen we met \texttt{pip}.

\texttt{pip} is de package manager van Python.

Gebruik de list optie om een lijst te krijgen van ge\"installeerde packages:
\begin{lstlisting}[language=bash]
$ pip list
\end{lstlisting}

Om meer te weten te komen van een package gebruiken we de show optie:
\begin{lstlisting}[language=bash]
$ pip show pip
\end{lstlisting}

Met de install en uninstall opties kunnen we packages installeren en de-installeren. In de opdrachten gaan we hiermee werken.

