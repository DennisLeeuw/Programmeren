Om characters om te zetten naar een binaire code nemen we een character, zoeken deze op in te tabel. De kolom geeft de eerste 3 bits en de rij geeft de tweede 4 bits, zo komen we op 7-bits:
\begin{description}
\item[A] 100 0001 = 65
\item[a] 110 0001 = 97
\item[0] 011 0000 = 48+0 = 48
\item[9] 011 1001 = 48+9 = 57
\end{description}
Het alfabet is op volgorde, dus B is 66 en b is 98 etc.


