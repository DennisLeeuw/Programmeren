\documentclass[a4paper,12pt]{article}
\usepackage[dutch]{babel}
\usepackage[utf8]{inputenc}
\usepackage{listings}
\usepackage{geometry}
\geometry{margin=2.5cm}

\title{Uitleg over de CSV-module in Python}
\author{}
\date{}

\begin{document}
\maketitle

\section{Wat is de csv-module?}
De \texttt{csv}-module in Python wordt gebruikt om CSV-bestanden (Comma Separated Values) te lezen en te schrijven. 
Een CSV-bestand is een tekstbestand waarin gegevens in tabellen worden opgeslagen — elke regel is een rij, 
en kolommen zijn gescheiden door een komma (of ander scheidingsteken zoals \texttt{;} of \texttt{\textbackslash t}).

\subsection*{Voorbeeld van een CSV-bestand}
\begin{lstlisting}[language={},frame=single]
naam,leeftijd,stad
Anna,25,Amsterdam
Bram,30,Rotterdam
\end{lstlisting}

\section{CSV-bestanden lezen}

\subsection{Met csv.reader()}
Leest het bestand regel voor regel en geeft elke rij terug als een lijst:
\begin{lstlisting}[language=Python,frame=single]
import csv

with open('personen.csv', newline='', encoding='utf-8') as csvfile:
    reader = csv.reader(csvfile)
    for rij in reader:
        print(rij)
\end{lstlisting}

\subsection{Met csv.DictReader()}
Leest het bestand en gebruikt de eerste rij als kolomnamen, 
waardoor elke rij een dictionary wordt:
\begin{lstlisting}[language=Python,frame=single]
import csv

with open('personen.csv', newline='', encoding='utf-8') as csvfile:
    reader = csv.DictReader(csvfile)
    for rij in reader:
        print(rij['naam'], rij['stad'])
\end{lstlisting}

\section{CSV-bestanden schrijven}

\subsection{Met csv.writer()}
\begin{lstlisting}[language=Python,frame=single]
import csv

with open('personen.csv', 'w', newline='', encoding='utf-8') as csvfile:
    writer = csv.writer(csvfile)
    writer.writerow(['naam', 'leeftijd', 'stad'])
    writer.writerow(['Anna', 25, 'Amsterdam'])
    writer.writerow(['Bram', 30, 'Rotterdam'])
\end{lstlisting}

\subsection{Met csv.DictWriter()}
\begin{lstlisting}[language=Python,frame=single]
import csv

with open('personen.csv', 'w', newline='', encoding='utf-8') as csvfile:
    veldnamen = ['naam', 'leeftijd', 'stad']
    writer = csv.DictWriter(csvfile, fieldnames=veldnamen)

    writer.writeheader()
    writer.writerow({'naam': 'Anna', 'leeftijd': 25, 'stad': 'Amsterdam'})
    writer.writerow({'naam': 'Bram', 'leeftijd': 30, 'stad': 'Rotterdam'})
\end{lstlisting}

\section{Belangrijke parameters}
\begin{itemize}
  \item \texttt{delimiter} — scheidingsteken (standaard ,)
  \item \texttt{quotechar} — teken om tekst te omgeven (standaard ")
  \item \texttt{quoting} — manier waarop waarden tussen aanhalingstekens worden gezet
\end{itemize}

\section{Samenvatting}
\begin{itemize}
  \item \texttt{csv.reader()} — leest lijsten
  \item \texttt{csv.DictReader()} — leest dictionaries
  \item \texttt{csv.writer()} — schrijft lijsten
  \item \texttt{csv.DictWriter()} — schrijft dictionaries
\end{itemize}

\end{document}
