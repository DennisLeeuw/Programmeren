Op basis van de index kan je ook een deel (range) van een lijst selecteren.
\begin{lstlisting}[language=python]
fruit = ["appel", "peer", "banaan", "sinaasappel", "kiwi"]
print(fruit[1:3])
\end{lstlisting}
Let op dat de range 1 tot 3 is, dus niet 1 tot en met 3. Je krijgt de elementen terug op posities 1 en 2.

Je kan ook een stuk vanaf het begin selecteren
\begin{lstlisting}[language=python]
fruit = ["appel", "peer", "banaan", "sinaasappel", "kiwi"]
print(fruit[:3])
\end{lstlisting}
Dit is dus van index 0 tot index 3.

Vanaf het einde mag ook, of beter van een positie tot het einde:
\begin{lstlisting}[language=python]
fruit = ["appel", "peer", "banaan", "sinaasappel", "kiwi"]
print(fruit[1:])
\end{lstlisting}

Met deze selecties kan je ook waarden wijzigen:
\begin{lstlisting}[language=python]
fruit = ["appel", "peer", "banaan", "sinaasappel", "kiwi"]
fruit[1:4] = [ "mango", "ananas", "papaya" ]
print(fruit)
\end{lstlisting}
De elementen op index 1 tot en met 3 worden overschreven met hun nieuwe waarde.

Als je meer elementen opgeeft als vervanging dan de opgegeven range groot is dan worden de extra elementen ingevoegd:
\begin{lstlisting}[language=python]
fruit = ["appel", "peer", "banaan", "sinaasappel", "kiwi"]
fruit[1:1] = [ "mango", "ananas", "papaya" ]
print(fruit)
\end{lstlisting}
We selecteren een range van geen elementen (van 1 tot 1), we kiezen dus alleen de positie van het eerste element en voegen dan 3 nieuwe elementen in.

Je kan een list ook korter maken:
\begin{lstlisting}[language=python]
fruit = ["appel", "peer", "banaan", "sinaasappel", "kiwi"]
fruit[1:4] = [ "mango" ]
print(fruit)
\end{lstlisting}
Alle elementen op index 1 tot en met 3 worden vervangen door 1 nieuw element.

Of \'e\`en of meer elementen verwijderen door hun waarde leeg te maken:
\begin{lstlisting}[language=python]
fruit = ["appel", "peer", "banaan", "sinaasappel", "kiwi"]
fruit[1:3] = ''
print(fruit)
\end{lstlisting}

