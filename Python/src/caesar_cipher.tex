Julius Caesar gebruikte encryptie om zijn orders door te geven aan zijn ondergeschikten. De reden dat hij encryptie gebruikte was om te zorgen dat zijn orders geheim beleven. Als de persoon die het bericht moest overbrengen gevangen werd genomen of werd gedood dan had de vijand alleen het ge-encrypte bericht en kende dus niet meteen de opdracht die uitgevoerd moest worden.

De techniek die Julius Caesar gebruikte was dat hij de letters van zijn bericht 1 of meer posities op schoof in het alphabet, dus een a werd dan bijvoorbeeld een d en een b werd dan een e. Door dit op een vaste manier te doen, konden de ontvangers het bericht relatief makkelijk decoderen, terwijl die vijand die niet wist hoevaak een letter verschoven was, als ze al wisten dat het om een verschuiving ging, niets met het bericht konden. In de tijd dat Julius Caesar deze techniek gebruikte, ongeveer 58 voor Christus, waren er sowieso niet veel mensen die konden lezen.

Stel dat we een bericht hebben met de tekst: 'Dit is een geheim bericht' en we schuiven alle letters één plaats op in het alfabet dan wordt het ge-encrypte bericht 'Eju jt ffo hfjn cfsjdiu'. Dat is al behoorlijk onleesbaar. Als we ook nog de spaties verwijderen 'Ejujtffohfjncfsjdiu' dan is het totaal onleesbaar geworden.

Later werd deze techniek bekend onder de naam: Caesar Cipher

Extra informatie over het Caesar Cipher:
\begin{itemize}
	\item \url{https://www.splunk.com/en\_us/blog/learn/caesar-cipher.html}
	\item \url{https://nl.wikipedia.org/wiki/Caesarcijfer}
\end{itemize}

