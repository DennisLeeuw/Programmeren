Gebruik \texttt{Putty} of \texttt{telnet} voor de volgende tests, denk om de extra lege regel na elk commando:

\begin{enumerate}
\item
\begin{lstlisting}[language=bash]
GET / HTTP/1.1

\end{lstlisting}

\item
\begin{lstlisting}[language=bash]
GET /ditbestaatniet HTTP/1.1

\end{lstlisting}

\item
\begin{lstlisting}[language=bash]
GET / HTTP/1.1
Host: google.com

\end{lstlisting}
\end{enumerate}

We zien bij de laatste opdracht dat de webserver niet kijkt naar de Host-header. De Python module voor de webserver is dan ook een simpele server. Maar leuk om wat basis zaken te testen.

Maak in de directory waarin je de webserver hebt gestart een eenvoudige website aan en start je browser op. Type in de tekstbalk \url{http://localhost:8080} en bekijk de headers die gebruikt zijn door de browser en welke er terug zijn gekomen van de server.

