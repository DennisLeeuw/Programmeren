Met vergelijkingsoperators\index{vergelijkingsoperators} (comparison operators\index{comparison operators}) kunnen twee waarden met elkaar vergelijken\index{vergelijken}. Het vergelijken noemen we het testen op een conditie, waarbij er waar\index{waar} (True\index{True}) of niet-waar\index{niet-waar} (False\index{False}) uit de vergelijking komt. In tabel \ref{table:compop} vind je de verschillende manieren waarop je een waarde kan vergelijken:
\begin{flushleft}
\begin{table}[h!]
\centering
	\begin{tabularx}{\textwidth}{ |c|X|c| }
\hline
	Operator &
	Betekenis &
	Hoe te gebruiken \\
\hline
	==\index{operator!==} &
	is gelijk aan\index{gelijk aan} &
	x == y \\
\hline
	!=\index{operator!"!=} &
	is niet gelijk aan\index{niet gelijk aan} &
	x != y 	\\
\hline
	\textgreater\index{operator!>} &
	is groter dan\index{groter dan} &
	x \textgreater y \\
\hline
	\textless\index{operator!<} &
	is kleiner dan\index{kleiner dan} &
	x \textless y \\
\hline
	\textgreater=\index{operator!>=} &
	is groter of gelijk aan\index{groter of gelijk aan} &
	x \textgreater= y \\
\hline
	\textless=\index{operator!<=} &
	is kleiner of gelijk aan\index{kleiner of gelijk aan} &
	x \textless= y \\
\hline
\end{tabularx}
\caption{Comparison operators}
\label{table:compop}
\end{table}
\end{flushleft}

