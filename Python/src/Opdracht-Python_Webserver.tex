\documentclass[a4paper,12pt,twoside,openright,titlepage]{article}

%Additional packages
\usepackage[utf8]{inputenc}
\usepackage[T1]{fontenc}
\usepackage[dutch,english]{babel}
\usepackage{syntonly}
\usepackage{csquotes}

% Symbols
\usepackage[official]{eurosym}
\usepackage{menukeys}

% Handle images
\usepackage{graphicx}
\graphicspath{ {./images/}{./styles/} }
\usepackage{float}
\usepackage{wrapfig}

% Handle URLs
\usepackage{xurl}
\usepackage{hyperref}
\hypersetup{colorlinks=true, linkcolor=blue, citecolor=blue, filecolor=blue, urlcolor=blue, pdftitle=, pdfauthor=, pdfsubject=, pdfkeywords=}

% Tables and listings
\usepackage{multirow,tabularx}
\usepackage[table]{xcolor} % Table colors
\usepackage{scrextend}
\addtokomafont{labelinglabel}{\sffamily}
\usepackage{listings}
\usepackage{adjustbox}

% Turn on indexing
\usepackage{imakeidx}
\makeindex[intoc]

% Define colors
\usepackage{color}
\definecolor{ashgrey}{rgb}{0.7, 0.75, 0.71}




% Listing style
\input{styles/lstset}

% Uncomment for production
% \syntaxonly

% Style
\pagestyle{headings}

% Define document
\author{D. Leeuw}
\title{Opdracht Python Webserver}
\date{\today\\
0.1.0 \\
\vfill
\raggedright
\copyright\ 2025 Dennis Leeuw\\
\input{styles/licentie-titlepage}}


\begin{document}
\selectlanguage{dutch}

\maketitle

%%%%%%%%%%%%%%%%%%%
%%% Introductie %%%
%%%%%%%%%%%%%%%%%%%

%B% \frontmatter
\section{Over dit Document}
\subsection{Leerdoelen}
Uit het Kwalificatie Dossier, P2-K1 Ontwikkelt digitale informatievoorzieningen:
\begin{itemize}
\item Heeft kennis van \'e\'en of meer programmeer- of scripttalen, zoals C\#, Java, Javascript, C++, Python, PHP, Bash of PowerShell
\item Heeft kennis van \'e\'en of meerdere dataformaten, zoals JSON, YAML, XML
\item Kan data ophalen uit of verzenden naar verschillende platformen, diensten of apparaten door gebruik van REST APIs
\end{itemize}


\subsection{Voorkennis}
Voor het uitvoeren van deze opdracht heeft de lezer de volgende voorkennis nodig:
\begin{itemize}
\item Windows gebruikers: Ervaring met Putty
\item Linux en Mac OS X gebruikers: Weten hoe \texttt{telnet} werkt
\end{itemize}



% Lessen:
% - Werken met HTTP 
%	- enable telnet
%	- telnet server 80
% 	- disable telnet
% - Werken met FastAPI en cURL
% - CTRL-C om de server te stoppen

%%%%%%%%%%%%%%%%%
%%% De inhoud %%%
%%%%%%%%%%%%%%%%%

\section{Een webserver starten}
Start je persoonlijke webserver
\begin{lstlisting}[language=bash]
python.exe -m http.server 8080
\end{lstlisting}

Start je browser op en type als URL \url{http://127.0.0.1:8080/} om te testen of de server werkt.

Maak in de directory waarin je de webserver hebt opgestart een \texttt{index.html} pagina aan met wat inhoud. Test in je browser.



\section{Spelen met een webserver}
Gebruik \texttt{Putty} of \texttt{telnet} voor de volgende tests, denk om de extra lege regel na elk commando:

\begin{enumerate}
\item
\begin{lstlisting}[language=bash]
GET / HTTP/1.1

\end{lstlisting}

\item
\begin{lstlisting}[language=bash]
GET /ditbestaatniet HTTP/1.1

\end{lstlisting}

\item
\begin{lstlisting}[language=bash]
GET / HTTP/1.1
Host: google.com

\end{lstlisting}
\end{enumerate}

We zien bij de laatste opdracht dat de webserver niet kijkt naar de Host-header. De Python module voor de webserver is dan ook een simpele server. Maar leuk om wat basis zaken te testen.

Maak in de directory waarin je de webserver hebt gestart een eenvoudige website aan en start je browser op. Type in de tekstbalk \url{http://localhost:8080} en bekijk de headers die gebruikt zijn door de browser en welke er terug zijn gekomen van de server.




%%%%%%%%%%%%%%%%%%%%%
%%% Index and End %%%
%%%%%%%%%%%%%%%%%%%%%
%\backmatter
\printindex
\end{document}

%%% Last line %%%
