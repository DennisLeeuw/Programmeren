Een webhosting bedrijf wil van zijn servers kunnen nagaan of alles nog correct werkt. Daarvoor moeten er een aantal tests gedaan worden:
\begin{enumerate}
	\item Controle of de server bereikbaar is
	\item Controle of de webserver draait
	\item Controle of de domeinen op de webserver beschikbaar zijn
\end{enumerate}
We mogen niets op de webserver installeren, dus we zullen dat vanaf afstand moeten doen.

De testen die we gaan doen moeten oplopen van algemeen naar specifiek:
\begin{enumerate}
\item ping de host
\item test of port 80 (HTTP) of port 443 (HTTPs) open staat
\item test of de server een bepaald domein serveert
\end{enumerate}
Als \'e\'en van de tests niet slaagt moet het script stoppen en een melding aan de gebruiker terug geven met welke test gelukt is en welke test er niet gelukt is.

De tests moeten in het programma dat je inlevert functies worden. Waarbij:
\begin{itemize}
	\item er bij de ping test een IP-adres als parameter verwacht wordt
	\item er bij de port test een IP-adres en een portnummer als parameters verwacht worden
	\item er bij de domein test een IP-adres, een portnummer en een domein als parameters verwacht worden
\end{itemize}

In het programma:
\begin{itemize}
	\item moeten er variabelen gebruikt worden voor IP, portnummer en domeinnaam
	\item moeten er try/except blokken gebruikt worden om fouten af te vangen
	\item moet er een for of while loop gebruikt worden voor het testen van de web-poorten
\end{itemize}

Opdracht 1
\begin{enumerate}
	\item Maak een flow-diagram van het te maken programmma
\end{enumerate}

Opdracht 2
\begin{enumerate}
	\item Schrijf een programma dat deze functies uitvoert
	\item Test het programma tegen een werkende webserver
\end{enumerate}

Opdracht 3
\begin{enumerate}
	\item Maak een versie waarin een gebruiker van het programma zelf mag opgeven op welk IP-adres, en domein er getest moet worden
	\item Het script per server uitzoekt of het port 443 of port 80 moet gebruiken
\end{enumerate}

