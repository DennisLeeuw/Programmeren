Syntax errors\index{syntax errors} zijn waarschijnlijk de meest voorkomende errors die je tegen komt als je een programmeertaal aan het leren bent. Een syntax error betekent dat je code geschreven hebt die zich niet aan de regels van Python houdt. Vaak zijn het kleine simpele dingen als het vergeten van een : of niet juist inspringen.

Een syntax error wordt in Python meestal duidelijk aangegeven als je de code probeert te runnen:
\begin{lstlisting}[language=python]
  File "/home/Boeken/Programmeren/Python/code/syntaxerror.py", line 1
    while True print('Hello world')
               ^^^^^
SyntaxError: invalid syntax
\end{lstlisting}
Met de carets (circumflex) (\textasciicircum) geeft Python aan waar vermoedelijk de fout zit. In dit geval moet er natuurlijk na de \texttt{True} een : komen en dus geeft Python aan dat we hier te maken hebben met een syntax error.

Omdat syntax errors fouten zijn die wij zelf gemaakt hebben moeten wij ze ook oplossen door de code aan te passen zodat deze wel voldoet aan de regels van Python.

Alle syntax errors kunnen we uit onze code halen door deze te testen (runnen). Deze fouten zouden er dus allemaal uit kunnen zijn voordat we het script daadwerkelijk in \textquote{productie} gaan gebruiken.

