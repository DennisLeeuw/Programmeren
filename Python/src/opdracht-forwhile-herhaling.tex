Maak een document met daarin de programma-code van de uitgewerkte opdrachten en een screenshot van de output van het programma. Leg uit wat de code doet en waarom.

\begin{enumerate}
\item Tafel van een getal
	\begin{description}
	\item[Opdracht] Vraag de gebruiker om een getal, en toon de tafel van dat getal van 1 t/m 10
	\end{description}

\item Aftellen
	\begin{description}
	\item[Opdracht] Laat de gebruiker een startgetal invoeren, en tel dan af tot 0 met een `while` loop.
	\end{description}

\item Gemiddelde berekenen
	\begin{description}
	\item[Opdracht] Vraag de gebruiker om 5 cijfers (bijv. van toetsen), en bereken het gemiddelde.
	\end{description}

\item Wachtwoord raden
	\begin{description}
	\item[Opdracht] Laat de gebruiker een wachtwoord invoeren. Blijf vragen tot het wachtwoord "geheim" is.
	\end{description}

\item Even of oneven
	\begin{description}
	\item[Opdracht] Toon of een reeks getallen van 1 t/m 20 even of oneven is.
	\end{description}

\item Maximaal gewicht
	\begin{description}
	\item[Opdracht] Je hebt een doos die maximaal 100 kg mag wegen. Vraag telkens om het gewicht van een voorwerp en tel dit bij elkaar op, tot de doos vol is (of eroverheen gaat). Geef aan hoeveel items er in de doos zitten en wat de doos weegt. Let op de doos mag niet stuk gaan! Voor die tijd moet de gebruiker gewaarschuwd worden.
	\end{description}

\end{enumerate}
