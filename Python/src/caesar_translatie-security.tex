We hebben gezien dat de reeks nummers voor uppercase letters een aparte reeks is van de reeks met lowercase letters. Daar moeten we rekening mee houden. Ook als we bijvoorbeeld de letter z \'e\'en plaats opschuiven dan moet het de a worden en niet \{ zoals dat volgens de ASCII-tabel zou worden. Om te weten te komen of een variabele uppercase of lowercase is kunnen we \texttt{islower} en \texttt{isupper} gebruiken. Voor het detecteren of een variabele een cijfer is is er \texttt{isnumeric}. Dat ziet er in Python code zo uit:
\begin{lstlisting}[language=python]
chars = [ 'a', 'A', '1', 'B', 'b', '2' ]
for char in chars:
    if char.islower():
        print(f"{char} is lowercase")
    elif char.isupper():
        print(f"{char} is uppercase")
    elif char.isnumeric():
        print(f"{char} is numeric")
\end{lstlisting}

Maak een script dat een substitutie uitvoert (encryptie). Gebruik een variabele voor het aantal keer dat er geschoven moet worden in het alfabet, geef deze variabele een waarde tussen 1 en 9 om de code niet te complex te maken. Denk erom dat je na de Z weer bij A moet beginnen, na z weer bij a en na 9 komt weer 0. Leestekens veranderen niet.

