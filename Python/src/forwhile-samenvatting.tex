Het belangrijkste verschil tussen een for- en while-loop is dat de for-loop kan worden gebruikt als het aantal iteraties bekend is en de while-loop kan worden gebruikt als het aantal iteraties niet bekend is.

\begin{table}[ht]
\centering
%\rowcolors{2}{gray!10}{gray!20}
\begin{tabular}{ |p{0.2\linewidth}|p{0.4\linewidth}|p{0.4\linewidth}| }
\hline
\rowcolor{gray!60}
& for-loop & while-loop \\
\hline
Loop variabele & Gedefinieerd in de loop aan het begin & Gedefinieerd buiten de loop, moet expliciet gedaan worden \\
\hline
Conditie & Controle voor elke iteratie & Controle voor elke iteratie \\
\hline
Update & Gedaan na elke iteratie & Gedaan in de loop, moet expliciet gedaan worden \\
\hline
Scope & Wordt bepaald door de loop body (voor gedefinieerd) & Moet expliciet gedaan worden \\
\hline
Gebruik & Als het aantal iteraties bekend is & Als het aantal iteraties niet bekend is, of als er aan een bepaalde conditie voldaan moet worden \\
\hline
\end{tabular}
\end{table}

