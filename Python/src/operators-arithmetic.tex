Operators\index{operators} worden gebruikt om acties (operaties) uit te voeren op variabelen en waarden. Een simpel voorbeeld:
\begin{lstlisting}[language=python]
print(42+3)
\end{lstlisting}
In dit voorbeeld is \texttt(+) de operator. Het vertelt de computer dat er iets moet gebeuren, namelijk optellen.

De Arithmetic operators\index{Arithmetic operators} zijn de operators voor rekenkundige bewerkingen. Python kent devolgende arithmetic operators:
\begin{flushleft}
\begin{table}[h!]
\centering
	\begin{tabularx}{\textwidth}{ |c|X|X| }
\hline
	Operator &
	Betekenis &
	Hoe te gebruiken \\
\hline
	+\index{operator!+} &
	optellen\index{optellen} van waarden;\newline samenvoegen van strings &
		6+5 \newline
	"Hello " + "World" \\
\hline
	-\index{operator!-} &
	aftrekken\index{aftrekken} van waarden &
	6-5 	\\
\hline
	*\index{operator!*} &
	vermenigvuldigen\index{vermenigvuldigen} van waarden &
	6*5 \\
\hline
	/\index{operator!/} &
	delen\index{delen} van waarden &
	6/5 \\
\hline
	**\index{operator!**} &
	machtensverheffen\index{machtsverheffen} &
	6**5 \\
\hline
	\%\index{operator!\%} &
	modulo rekenen\index{modulo rekenen} &
	14\%12 \\
\hline
	//\index{operator!//} &
		geheeltallige deling\index{geheeltallige deling} (floor devision\index{floor devision}) &
	14//12 \\
\hline
\end{tabularx}
\caption{Comparison operators}
\label{table:compop}
\end{table}
\end{flushleft}

