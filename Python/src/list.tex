Een list\index{list} of zoals het in andere programmeertalen wel wordt genoemd een array\index{array} is een lijst van waarden die aan \'e\'en variable is toegekend. Voorbeeld:
\begin{lstlisting}[language=python]
fruit = ["appel", "peer", "banaan", "sinaasappel"]
print(fruit)
\end{lstlisting}
Een list wordt gedefinieerd door de blockhaken met daartussen de waarden die er in de list zitten gescheiden door komma's.

We kunnen in Python verschillende variabelen vullen door er een lijst met waarden aan te geven. Dan kunnen we ook doen met een list:
\begin{lstlisting}[language=python]
fruit = ["appel", "peer", "banaan", "sinaasappel"]
n, m, o, p = fruit
print(n)
print(m)
print(o)
print(p)
\end{lstlisting}

Een list is een ge\"indexeerde lijst. Het eerste item staat op plek 0, het tweede op plek 1, etc. Met print kan je dat ook zien:
\begin{lstlisting}[language=python]
fruit = ["appel", "peer", "banaan", "sinaasappel"]
print(fruit[0])
print(fruit[2])
\end{lstlisting}
Een index mag ook negatief zijn, dan is -1 het laatste element van de lijst en -2 de voorlaatste.
\begin{lstlisting}[language=python]
fruit = ["appel", "peer", "banaan", "sinaasappel"]
print(fruit[-1])
print(fruit[-3])
\end{lstlisting}

Omdat een list een ge\"indexeerde lijst is kunnen er ook dubbele waarden voor komen:
\begin{lstlisting}[language=python]
fruit = ["appel", "peer", "banaan", "sinaasappel", "banaan"]
print(fruit)
\end{lstlisting}
We kunnen op basis van de index ook een waarde wijzigen:
\begin{lstlisting}[language=python]
fruit = ["appel", "peer", "banaan", "sinaasappel", "banaan"]
print(fruit)
fruit[2] = "kiwi"
print(fruit)
\end{lstlisting}
De eerste banaan in de lijst is nu een kiwi geworden.

Samengevat is een list dus geordend, staat hij duplicaten toe en kunnen de elementen gewijzigd worden.

