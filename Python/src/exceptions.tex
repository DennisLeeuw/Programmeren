Exceptions\index{exceptions} zijn runtime errors\index{runtime errors} ontstaan doordat er iets in de software gebeurd dat we niet hebben voorzien. Deze fouten worden pas gevonden door de software te gebruiken. Runtime errors hoeven geen fouten te zijn waarop de software crashed. We kunnen ze afvangen met een zogenaamd \texttt{try}\index{try} / \texttt{except}\index{except} block. Dat ziet er zo uit:
\begin{lstlisting}[language=python]
try:
    resultaat = 10 / 0
except ZeroDivisionError:
    print("Delen door nul is niet toegestaan.")
\end{lstlisting}

Onder het kopje \texttt{try} wordt een actie uitgevoerd en bij het kopje \texttt{except} wordt de fout afgehandeld. De fout afhandeling kan een simpele print zijn zoals in het voorgaande voorbeeld waarbij je de gebruiker vertelt wat er fout gegaan is.

Natuurlijk zou je zelf nooit door 0 delen, maar het kan zijn dat er een variabele is die in sommige situaties 0 wordt en dat daarna deze fout ontstaat. Zeker als een programma veel gebruik maakt van gebruikers invoer of data van anderen die via een bestand wordt ingelezen, dan kunnen er soms waarden in variabelen terecht komen die je niet zou verwachten. Zo kan er dan een fout situatie ontstaan die je vooraf niet bedacht had.

De melding \textquote{Je kunt niet delen door nul} is ook geen zinnige meldig. We zouden hier meer informatie naar de gebruiker terug kunnen geven. Bijvoorbeeld om welke variabele het gaat, of wat de functie van de deling is. Het goed schrijven van een error-melding is vaak nog een hele kunst, omdat je ook niet altijd weet de technische kennis is van de gebruiker.

Tijdens de werking van een programma kunnen er verschillende fouten ontstaan die afgevangen kunnen worden. Voor een overzicht van de mogelijke fouten en hun naam (except) is er een lijst van Built in exceptions\index{built in exceptions}\index{exceptions!built in} op W3Schools: \url{https://www.w3schools.com/python/python_ref_exceptions.asp}. Lees deze lijst door zodat je weet welke fouten je met try/except standaard kunt afvangen. Niet alle exceptions zullen meteen duidelijk zijn, maar een aantal zou je direct kunnen gaan gebruiken.

Met try/except kunnen we bijvoorbeeld de input van een gebruiker controleren:
\begin{lstlisting}[language=python]
while True:
    try:
        x = int(input("Geef een nummer op: "))
        break
    except ValueError:
        print("Dat was geen nummer, probeer het nog een keer.")

print(f"De ingevoerde waarde is: {x}")
\end{lstlisting}

Dit werkt zo:
\begin{enumerate}
	\item De \texttt{while}-loop loopt door tot de \texttt{break} uitgevoerd kan worden. Daarna krijgt de gebruiker de melding welke waarde er opgegeven is.
	\item De \texttt{try} vraagt aan de gebruiker om een getal in te voeren en dit getal moet een integer (int) zijn. Als dit niet zo is dan is er een error, een \texttt{ValueError}. Een error breekt de \texttt{try} meteen af zodat deze nooit bij de \texttt{break} komt.
	\item De \texttt{except} vangt de error af en geeft een melding naar de gebruiker met wat deze fout gedaan heeft. De \texttt{while} zorgt ervoor dat we meteen weer bij de \texttt{try} terecht komen.
\end{enumerate}

We kunnen de twee voorbeelden ook combineren, waarbij we \'e\'en \texttt{try} hebben met meerdere exceptions, namelijk: \texttt{ValueError} en \texttt{ZeroDivisionError}:
\begin{lstlisting}[language=python]
try:
    x = int(input("Geef een nummer op: "))
    resultaat = 10 / x
    print("Resultaat is: ", resultaat)
except ValueError:
    print("Dat was geen nummer, probeer het nog een keer.")
except ZeroDivisionError:
    print("Delen door nul is niet toegestaan.")
\end{lstlisting}

Tot slot is er nog de algemene exception genaamd \texttt{Exception}. Die kunnen we gebruiken als een laatste red middel:
\begin{lstlisting}[language=python]
try:
    x = int(input("Geef een nummer op: "))
    resultaat = 10 / x
    print("Resultaat is: ", resultaat)
except ValueError:
    print("Dat was geen nummer, probeer het nog een keer.")
except ZeroDivisionError:
    print("Delen door nul is niet toegestaan.")
except Exception as error:
    print("Er is iets fout gegaan. Het systeem meldde: ", error)
\end{lstlisting}
Hier geven de door Python gemeldde exception in de \texttt{except} door als waarde in de variabele \texttt{error}. En deze \texttt{error} variabele kunnen we dan weer op het scherm afbeelden in de hoop dat de gebruiker of de programmeur er iets mee kan.

