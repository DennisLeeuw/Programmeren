Tuurlijk! Hier zijn 5 Python-opdrachten op MBO niveau 4, allemaal zonder functies, en met gebruik van een `for` of `while` loop. Ze zijn ideaal voor oefening met herhalingen en basislogica.

---

### 🐍 **Opdracht 1: Tafel van een getal**
**Omschrijving:** Vraag de gebruiker om een getal, en toon de tafel van dat getal van 1 t/m 10.

```python
getal = int(input("Van welk getal wil je de tafel zien? "))

for i in range(1, 11):
    print(f"{getal} x {i} = {getal * i}")
```

---

### 🔢 **Opdracht 2: Aftellen**
**Omschrijving:** Laat de gebruiker een startgetal invoeren, en tel dan af tot 0 met een `while` loop.

```python
start = int(input("Voer een startgetal in: "))

while start >= 0:
    print(start)
    start -= 1
```

---

### 📊 **Opdracht 3: Gemiddelde berekenen**
**Omschrijving:** Vraag de gebruiker om 5 cijfers (bijv. van toetsen), en bereken het gemiddelde.

```python
totaal = 0

for i in range(5):
    cijfer = float(input(f"Voer cijfer {i+1} in: "))
    totaal += cijfer

gemiddelde = totaal / 5
print(f"Het gemiddelde is: {gemiddelde}")
```

---

### ✅ **Opdracht 4: Wachtwoord herhalen**
**Omschrijving:** Laat de gebruiker een wachtwoord invoeren. Blijf vragen tot het wachtwoord "geheim" is.

```python
wachtwoord = ""

while wachtwoord != "geheim":
    wachtwoord = input("Voer het wachtwoord in: ")

print("Wachtwoord correct!")
```

---

### 🎲 **Opdracht 5: Getallen raden**
**Omschrijving:** De gebruiker moet een getal tussen 1 en 10 raden. De juiste waarde is 7. De loop stopt pas als hij goed geraden is.

```python
antwoord = 0

while antwoord != 7:
    antwoord = int(input("Raad een getal tussen 1 en 10: "))
    if antwoord != 7:
        print("Helaas, probeer opnieuw.")

print("Goed geraden!")
```

---

Laat me weten als je er meer wilt, of als je ze iets uitdagender of juist makkelijker wilt maken!
