## ✅ **Opdracht 1: Schijfruimte controleren**

### 🧪 Voorbeeldcode:
```python
import shutil

def controleer_schijfruimte():
    totale, gebruikt, vrij = shutil.disk_usage("C:/")
    vrij_gb = vrij / (1024 ** 3)

    if vrij_gb < 5:
        print(f"Waarschuwing: minder dan 5 GB beschikbaar! ({vrij_gb:.1f} GB)")
    else:
        print(f"Vrije ruimte op C:/ is {vrij_gb:.1f} GB – voldoende.")
        
# Test de functie
controleer_schijfruimte()
```

---

## ✅ **Opdracht 2: Automatisch gebruikers aanmaken (simulatie)**

### 🧪 Voorbeeldcode:
```python
def maak_gebruiker_aan(gebruikersnaam, wachtwoord="Welkom123"):
    print(f"Gebruiker '{gebruikersnaam}' succesvol aangemaakt.")
    print(f"Wachtwoord ingesteld op: {wachtwoord}")

# Test
maak_gebruiker_aan("student123")
maak_gebruiker_aan("admin", "Beheerder2025")
```

---

## ✅ **Opdracht 3: Logbestanden opschonen**

### 📁 Voorbeelddata:
Simuleer een map `C:/logs/` met bestanden zoals:
- `log_1.log` (ouder dan 30 dagen)
- `log_2.log` (nieuw)
- `info.txt`

### 🧪 Voorbeeldcode:
```python
import os
import time
from datetime import datetime, timedelta

def verwijder_oude_logs(pad):
    nu = time.time()
    grensdatum = nu - (30 * 24 * 60 * 60)  # 30 dagen in seconden

    for bestand in os.listdir(pad):
        if bestand.endswith(".log"):
            volledig_pad = os.path.join(pad, bestand)
            aanmaak_tijd = os.path.getmtime( volledig_pad )

            if aanmaak_tijd < grensdatum:
                os.remove( volledig_pad )
                print(f"Verwijderd: {bestand}")
            else:
                print(f"Bewaard: {bestand} (recent)")

# Test
verwijder_oude_logs("C:/logs")
```

📌 **Let op**: Test dit eerst met tijdelijke bestanden, zodat er niets belangrijks wordt verwijderd!

---

## ✅ **Opdracht 4: Systeemrapport genereren**

### 🧪 Voorbeeldcode:
```python
import psutil
import socket

def genereer_systeemrapport():
    ram = psutil.virtual_memory().total / (1024 ** 3)
    cpu_info = psutil.cpu_freq()
    ip = socket.gethostbyname(socket.gethostname())

    print("Systeemrapport:")
    print(f"RAM: {ram:.1f} GB")
    print(f"CPU snelheid: {cpu_info.current:.2f} MHz")
    print(f"IP-adres: {ip}")

# Test
genereer_systeemrapport()
```

🛠️ **Voor deze opdracht moet je soms `psutil` installeren** met:
```bash
pip install psutil
```

