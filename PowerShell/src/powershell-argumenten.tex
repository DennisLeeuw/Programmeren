Bij scripting is het handig als een gebruiker parameters aan een script kan meegeven om het script bepaalde acties te laten uitvoeren:
\begin{lstlisting}[language=bash]
.\mijnscript.ps1 -Naam "Jan" -Leeftijd 25
\end{lstlisting}

Via de \texttt{param} functie kunnen we afdwingen wat voor informatie er gegeven mag worden en in welke variabele de data geplaatst wordt:
\begin{lstlisting}[language=bash]
param (
    [string]$Naam,
    [int]$Leeftijd
)

Write-Host "Hallo $Naam, je bent $Leeftijd jaar oud."
\end{lstlisting}

We kunnen ook afdwingen dat er iets opgegeven moet worden (mandatory):
\begin{lstlisting}[language=bash]
param (
    [Parameter(Mandatory=$true)]
    [string]$Bestand
)
\end{lstlisting}

Om te voorkomen dat we met lege variabelen komen te zitten, kunnen we ook een default waarde opgeven:
\begin{lstlisting}[language=bash]
param (
    [string]$Naam = "Onbekend"
)
\end{lstlisting}

Ook kunnen we een lijst opgeven van geldige waarden, als niet \'e\'en van de waarden geven wordt dan geeft PowerShell een error.
\begin{lstlisting}[language=bash]
param (
    [ValidateSet("Start", "Stop", "Restart")]
    [string]$Actie
)
\end{lstlisting}

